\subsection{Objects as a proxy}
\label{sec:rendering-proxy}

Objects that need to be rendered to the screen act as a proxy for the 
RenderObject. The only thing those objects need to do to be rendered can be 
seen in \cref{lst:rendering}. As you can see, the objects only need to return 
the result of its own RenderObject. Objects that need custom rendering logic 
can have a rendering object that derives from the SDL\_RenderObject class seen 
in \cref{fig:renderobject-inherit}.
\\
\begin{lstlisting}[caption={Rendering proxy.},label={lst:rendering}]
void BaseEntity::render(SDLRenderer *renderer) {
    return representation->render(renderer);
}
\end{lstlisting}

When a RenderObject renders itself, it sends the texture it created to the 
SDLRenderer class to draw it on a back buffer. After the rendering loop is 
completed, the back buffer is drawn to the screen using the 
\lstinline{renderer->draw_to_back_buffer(SDL_Texture *, SDL_Rect *)} 
method shown in \cref{lst:drawtobackbuffer}.
\\

\begin{lstlisting}[caption={Drawing to the back buffer.},
label={lst:drawtobackbuffer}]
void SDLRenderer::draw_to_back_buffer(SDL_Texture *t, SDL_Rect *r) {
    SDL_SetRenderTarget(engine, _back_buffer);
    // Blend the textures
    SDL_SetTextureBlendMode(_back_buffer, SDL_BLENDMODE_BLEND);
    SDL_SetTextureBlendMode(t, SDL_BLENDMODE_BLEND);
    if (SDL_RenderCopy(engine, t, NULL, r) < 0) {
        std::cerr << SDL_GetError() << std::endl;
    }
    SDL_SetRenderTarget(engine, NULL);
}
\end{lstlisting}
