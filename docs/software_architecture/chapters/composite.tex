\subsection{Composite goals}
We created a couple of Composite goals for moving entities which we describe below. There is a class diagram \cref{fig:goalcomposite-inherit} in the appendix that illustrates the structure. This diagram doesn't show all composite goals, however it should give an impressions how we implemented this. 

\subsubsection{Think}
The Think goal is a goal that never gets removed. This goal is needed 
to determine an entity's next goal and activates that goal. It does so by 
calling the goals evaluator class, which returns a desirability value. The 
goal with the highest desirability gets chosen as the next goal. Desirability’s can be influenced by a variant of factors. E.g., how far away is the entity from an hostile entity. When there is an hostile enemy close it might not want to gather resources. Evaluators are a great way to create AI. Instead of using these evaluators we might use fuzzy logic. With fuzzy logic the actions of the entity should even feel more natural. 

\subsubsection{Follow Path}
\label{sec:followpath}
This goal traverses a path by adding the TraverseEdgeGoal class to its 
subgoals. This way a path consists of multiple instances of TraverseEdgeGoals
(\cref{sec:traverseedge}) which can be paused if the entity needs to do 
something else first, i.e. fleeing or resting. Once there are no more edges 
to traverse, this goal will be completed and removed from the entity's 
subgoals.

\subsubsection{Work}
This goal makes the entity go to the nearest resource to work. It first plans 
a path (\cref{sec:planpath}) to the nearest resource and then it follows that 
path (\cref{sec:followpath}). Once it arrives on the resource's location it 
starts to gather the resource (\cref{sec:gatherresource}). Once it's done 
gathering it plans another path, this time to the closest warehouse/depot. It follows 
this path and after it arrives it drops its resources 
(\cref{sec:dropresources}). Once all these goals are completed the work goal is completed as well.

