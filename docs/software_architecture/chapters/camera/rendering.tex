\subsection{Used in rendering}

When rendering, we need a way to determine what entities need to be drawn to 
the screen, since screen coordinates and world coordinates are not connected. 
This is where the camera's matrix comes into play, because this matrix 
transforms the world coordinates into screen coordinates.
\\
The CameraManager class has a method that accepts a reference to a different 
class method, almost like a callback function. This method passes the 
camera's matrix to the given method. This is done so that you can reuse this 
method and add functionality to it that you want to be run before calling the 
world's render method. The whole method looks like 
(\cref{lst:cameramanager-render}):
\\
\begin{lstlisting}[caption={CameraManager render method.},
label={lst:cameramanager-render}]
template<typename W, typename T, typename M>
void CameraManager::render(T *t, W *w, void (W::*f)(T *, const M &)) {
    (w->*f)(t, _camera->get_world_model());
}
\end{lstlisting}
~\\
The way to use this method is shown in \cref{lst:cameramanager-render-call}:
\\
\begin{lstlisting}[caption={Using the CameraManager::render method.},
label={lst:cameramanager-render-call}]
CameraManager::get_instance()->render<World, SDLRenderer, mat2>(renderer, _current_world, &World::render);
\end{lstlisting}
