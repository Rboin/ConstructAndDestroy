\section{Behaviour}
\label{sec:behaviour}
We've chosen to use the goal driven behavior approach, because entities need to complete different actions to complete a goal. E.g., for a lumberjack to collect wood it needs to plan a path to the resource, then it needs to follow the path, once arrived it should start gathering, etc. Goal driven behavior provides a solid solution for these types of actions. Goals can be very large with loads off sub goals or actions, or they can be very small, this makes goal driven behavior easier extendable compared to state driven behavior for example. Since a goal can consist of multiple smaller goals, the Composite Pattern is a good solution to this problem. You can have small goals such as 'TraverseEdge' and also bigger goals like 'Work' and still treat them the same way. \cite{composite-pattern}

We created a single base class called Goal, which is a template class so that we can reuse it for different entity types. This class, together with the AtomicGoal and CompositeGoal classes, are shown in \cref{fig:goal}. Besides the Goal<T> class, we created two classes that inherit from it, the AtomicGoal and GoalComposite classes. The GoalComposite class contains a deque data structure that contains its subgoals. As you can see in \cref{fig:goal}, the GoalComposite<T> class also contains a couple of extra methods to add, remove, process and remove subgoals.

\subsection{Composite goals} 
We created a couple of Composite goals for moving entities which we describe 
below. There is a class diagram \cref{fig:goalcomposite-inherit} in the 
appendix that illustrates the structure. This diagram doesn't show all 
composite goals, however it should give an impressions how we implemented this. 

\subsubsection{Think} 
The Think goal is a goal that never gets removed. This goal is needed to 
determine an entity's next goal and activates that goal. It does so by calling 
the goals evaluator class, which returns a desirability value. The goal with 
the highest desirability gets chosen as the next goal. Desirability’s can be 
influenced by a variant of factors. E.g., how far away is the entity from an 
hostile entity. When there is an hostile enemy close it might not want to 
gather resources. Evaluators are a great way to create AI. 

\subsubsection{Follow Path} 
\label{sec:followpath} 
This goal traverses a path by adding the TraverseEdgeGoal class to its subgoals. 
This way a path consists of multiple instances of TraverseEdgeGoals 
(\cref{sec:traverseedge}) which can be paused if the entity needs to do 
something else first, i.e. fleeing. Once there are no more edges to traverse, 
this goal will be completed and removed from the entity's subgoals.

\subsubsection{Work} 
This goal makes the entity go to the nearest resource to work. It first plans 
a path (\cref{sec:planpath}) to the nearest resource and then it follows that 
path (\cref{sec:followpath}). Once it arrives on the resource's location it 
starts to gather the resource (\cref{sec:gatherresource}). Once it's done 
gathering it plans another path, this time to the closest warehouse/depot. It 
follows this path and after it arrives it drops its resources 
(\cref{sec:dropresources}). Once all these goals are completed the work goal 
is completed as well.

\subsubsection{Combat} 
\label{sec:combat} 
This goal can be initialized with or without an enemy. When this goal is 
initialized with an enemy it means the owner of this goal is being attacked, 
thus it doesn’t have to hunt a target (\cref{sec:hunttarget}) and the fight 
goal can be added right away. When the goal is initialized without an enemy it 
means this entity will look for an enemy. If there is no enemy entity present 
the goal will be removed, if there is an enemy the goals hunt target and fight 
goal will be added to the owner.

\subsubsection{Hunt Target} 
\label{sec:hunttarget} 
This goal will hunt down an enemy moving entity. This is done by using the 
plan path (\cref{sec:planpath}) and follow path (\cref{sec:followpath}) goals. 
The most notable thing for this goal is when it is completed or terminated. If 
the sub goals of hunt target are empty it means we have followed the path to 
the enemy’s location. Since we are chasing moving entities we regularly have 
to check whether or not the enemy has moved to much from its initial location. 
When this is the case we set a seek to the current enemy location, after that 
the goal will be removed. We have to do this otherwise the entity will stop 
moving for a bit. After this the combat goal (\cref{sec:combat}) will be called 
again and the hunt target goal can plan a new path to the enemy’s location. 
This process is repeated untill we are close enough to attack the enemy entity.

\subsubsection{Wander} 
The wander goal is a goal that gets activated when there are no enemies 
present on the map. A random node is chosen and we plan a path to that node. 
Then the followpath goal will be activated and we traverse the edges. This 
process is continued until a new enemy enters the game world.


\subsection{Atomic goals}
Besides Composite goals, we also have Atomic goals. You can compare Atomic 
goals with leaf nodes of a tree structure. Atomic goals are the actual actions
 that an entity needs to do. As shown in \cref{fig:atomicgoal-inherit}, an atomic 
 goal has no subgoals. Calling the AtomicGoal::add\_subgoal() method results in an 
 exception.
 
 \begin{figure}[!htb]
    \centering
    \includegraphics{res/AtomicGoal-Inherit.jpg}
    \caption{AtomicGoal Inheritance.}\label{fig:atomicgoal-inherit}
\end{figure}

\subsubsection{Wander}
The wandering goal is another goal that never gets removed, just like the 
think goal. An entity always needs to wander around if it has absolutely 
nothing to do. The only thing this goal does is activate the wander steering 
behaviour (\cref{sec:wander}).

\subsubsection{Obstacle Avoidance}
The obstacle avoidance goal activates the obstacle avoidance steering 
behaviour. When this goal is activated, it adds the steering behaviour to the 
entity's behaviours. Once the distance to the target that it needs to avoid 
is big enough, the goal is completed and the steering behaviour also gets 
removed from the entity.

\subsubsection{Drop Resources}
\label{sec:dropresources}
Once an entity has gathered enough resources, it needs to drop them at a 
warehouse/depot. This goal simply removes the resources the entity gathered and adds them to the resources of the player. When it dropped all of it's resources at the warehouse, the goal is completed.

\subsubsection{Gather Resource}
\label{sec:gatherresource}
This goal calls the Gather() method from a resource entity. This method extracts resources from the resource entity and adds it to the entity that is gathering the resource. Once the resource entity has been depleted or the maximum carrying capacity of the gathering entity has been reached this goal will be completed.

\subsubsection{Plan Path}
\label{sec:planpath}
The plan path goal plans a path using the A* algorithm, using the Manhattan 
heuristic (\cref{sec:pathplanning}). Once the path has been generated, it gets set as the active path 
for the given entity. This is the only task it needs to complete before 
getting removed from the containing goal.

\subsubsection{Traverse Edge}
\label{sec:traverseedge}
This goal uses the Seek behaviour explained in \cref{sec:seek-behaviour}. Once 
this class gets instantiated it adds an instance of SeekStrategy to the 
entity's behaviour. The only thing it needs to do while processing, is 
to check whether the entity has this behaviour, and if so if it's close 
enough to the next node on the graph. If it's close enough, the goal has been 
completed.


\newpage

