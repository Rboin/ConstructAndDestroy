\section{Robin Koning}

I contributed mostly to the backend of our game. I set up the initial 
project, with the base- and movingentity, the rendering with SDL2 and the 
documentation with \LaTeX. I also wrote the initial framework for 
Goal-driven behaviours, including the Think goal, and some steering 
behaviours such as the Seek- and Arrivebehaviour. 

Secondly, I made sure to split our game into three different projects, 
namely the logic, rendering and UI projects. I did this to create a better 
overview of our project and to create a divide between gamelogic and 
rendering/ui. Besides the better overview, it also gave us a way to switch to 
OpenGL more easily, if we so desired, because the rendering is completely 
seperate from the actual game logic and UI. I'm very happy with this divide, 
even though I could've handled the updating of the other branches (on github) 
better.

I also made it possible to create UI elements, such as panels, buttons and 
text. SDL2 is just a simple rendering library which supplies 
you with a window handle and a renderer, so I came up with a way for us to 
create these UI elements. Besides the UI elements, I also wrote the event 
system, which includes event dispatchers and a way to hook up a UI element 
to an event. The way the UI framework and event system turned out, I'm pretty 
happy with. It's really extensible because of the composite pattern used, and 
simple to use.
Also, I wrote a wrapper that handles the rendering to textures and the 
screen, with different back buffers for the UI and the world/entities. I 
think this simplified the rendering logic a lot, since you don't have to copy 
your own texture to the correct back buffer anymore. Instead, you can now 
send an image to the backbuffer or use SDL2's RenderDraw functions, and 
it will be applied to the correct back buffer.

The Camera, with corresponding manager class, is also written by me, the 
camera makes it possible to move around within the world, to zoom and it's 
also used for transforming world coordinates into screen coordinates. I'm 
very happy with how the camera turned out. First we just had a static world, 
which was just the initial screen, but now the screen and world coordinates 
are seperated and we can create larger maps.

I also wrote the wave system and corresponding panel with button. 
The creation of the panel, button and click event was really simple, because 
I have extensive knowledge about how our UI framework works, since I wrote it.
Making the wave system was a little bit more difficult, because I had to use 
multiple timers for the different stages of a wave and it was hard to keep 
track of those timers while programming.

Besides the material contributions that I have made, I also supported my 
teammates when they had difficulties using the UI or rendering frameworks or 
just C++ in general. I tried to help and give feedback when someone got 
stuck on a problem.

There were also some difficulties concerning memory leaks and Git.
About halfway through the project, we found out that we had some memory leaks.
It was pretty difficult finding out exactly where the leaks occurred, without 
using a memory monitoring tool like Valgrind.
Also, when I split the project into some smaller ones, we had a lot of 
difficulties applying those changes to the other branches. I could've handled
this better by creating smaller patches instead of one big patch.

Some personal qualities that I think I have shown during this project are:

\textbf{Helpful.} Whenever someone asked for my help I tried to do this as 
best as I could. If I didn't know how to resolve a problem immediately, I 
hopped on to the other's branch and tried to resolve the problem by looking 
through his programming and giving pointers on what could be wrong or how to 
proceed from the current state.

\textbf{Initiative.} If I found something in our code that could be improved 
I created an issue on GitHub. If it was something in the backend, like the 
camera and rendering, I assigned it to myself. There was also the splitting 
of our project into smaller ones, which I thought of as pretty high priority.
I assigned this to myself to do during the holidays.

\textbf{Self-Critical.} I had a pretty high standard for my own code quality.
If I wasn't satisfied with my own code or if it didn't work exactly how I 
wanted it to work; I was not afraid to throw it all away and start over again.

About halfway through the project, we found out that we had some memory leaks.
It was pretty difficult finding out exactly where the leaks occurred, without 
using a memory monitoring tool like Valgrind.

When I split the project into some smaller ones, we had a lot of difficulties 
applying those changes to the other branches. 

Some new goals for a future project are:

\textbf{Be more vigilant while reviewing}, some small issues still slipped 
through the cracks, like unnecessary includes or non-initialized members.

\textbf{Use Valgrind (or Dr. Memory) from the start while programming in C++.} 
This would lower the total count of uninitialized members and other memory 
related leaks. It would save a lot of time if I used it from the start, 
instead of searching through a lot of code before finding the actual source 
of a leak.
