\section{Jeroen Vinke}
During the first sprint I worked on a feature that lets a player order units to do something. For example, when the player selects a woodcutter and right clicks on a tree, the woodcutter will move to that tree and start gathering resources. When the player right clicks on a part of the ground, all selected entities will move to that location.

For the remaining two sprints I worked on many different panels. During the second sprint I made a control panel which consists of two other panels to spawn entities and buildings. When the player clicks on a building it would allow the player to place a building on the map and when the player clicks on an entity in the control panel the entity would spawn in the world.

The third (and last) sprint I made a unit information panel. This unit information panel can show statistics of a single entity, a single building or multiple entities. This is all based on the active selection of the player. 

Beside these activities I have also fixed many bugs, especially during the second sprint. For example, there was a bug which caused unit selection to sometimes fail. I was able to track down the cause to the Binary Search Partitioning tree and was able to identify the issue and solve it. I have also done plenty of code reviews.
 
The game is much nicer to play because of the panels that I have added to the game. It also gives the player insight into what is happening in the game. One thing that comes to mind is that the player can now actually see the amount of wood that woodcutters carry increase when they are working. It also allows the player to see exactly how much health and attack damage entities have. 

There have been two instances where I started writing documentation three days after I finished the code. I should have written the documentation earlier so that the feature could have been marked as done. This could have become an issue if we would have had a lot of work to do near the end of the sprint, but luckily this was not the case.

I always enjoy helping others out during a project, if possible. This project was no exception. There was one time when I helped Stephan to debug the resources panel, which had a bug where it showed that the player had -4 gold at some point, which never should have happened. Also, because I had worked on panels during the second sprint, I helped Mark to understand how the panels worked. This allowed him to finish his user story more quickly.

I have shown to be reliable when it comes to arrangements. When I told the group that I would write the progress report and hand it in, I would not forget.

With respect to my work, I believe it matches the average quality (and quantity) of the rest of the group. In the beginning the quality and quantity of my work was lower, because I wasn’t used to C++. A lot of time was spent on debugging error messages that I didn’t understand. In the beginning of the project I also missed a lot of destructors, which caused memory leaks which we had to fix later on. But this happened with other members of the team as well. I always kept an eye on performance and tried to optimize the code where it was possible. I also made sure to use design patterns when possible. So because of this I certainly think that the quality of my code was as good as the rest of my team.

There are a few things that I really like about this team. I believe that we managed to produce a lot. Especially in the last two sprints we were able to add a lot of features to the game. At the end of the first sprint we had a woodcutter that was able to move to a tree and cut wood, and at the end of the third sprint we had an entire game that was fun to play. That was cool.

Another thing that I like about this team is that we don’t argue too much, but not too little either. Even though we have a difference in opinions sometimes, we always seem to come to a conclusion and we move on. 

We also didn’t discuss every little detail of a technical solution of a feature that we wanted to add to the game. Instead we were able to discuss things on a functional level, which has allowed us to get great end results for each sprint.

Sometimes it’s necessary to organize things, especially when someone’s work depends on the work of someone else. We didn’t always discuss this in the group, so when something like that happened to me I always talked to that someone and made sure that we were on the same page. I’m sure that this helped to prevent some issues later on in the sprint.

I have learned that C++ has some quirks that you would only know of when you’re working on a project like this one. For example, I wanted to initialize a vector from a .h file, which caused many errors and I wasn’t able to make any sense of them. It turned out that this wasn’t allowed, and that it had to be initialized elsewhere. I also tried to remove an item from a vector when I had iterated over the vector using integers instead of pointers. This caused some very strange issues which were resolved by performing the removing action with a pointer instead.

Another thing I learned is that we work together really well as a team. I don’t think we have the opportunity to do another project together since we’re all doing different things next year, but if we would I’d gladly do it with this team again. 

If I ever have to work on a C++ project again then I would study the C++ compiler. I know the language for the most part but I don’t know a whole lot about the compiler. This makes it difficult to debug some of the error messages it outputs. It also causes you to encounter many ‘weird’ issues because you don’t understand what is happening.

We wrote our documentation with Latex. Before this project I had never heard of Latex, and was hesitant to use it because Word worked fine for me in the past. But after having looked into it I wanted to give it a try. The Latex editor we used isn’t my favorite, but Latex in general is something I like. One major advantage of Latex over Word is that it allows you to resolve merge conflicts with Git. This made it almost similar to the way we merge in code. I never learned how to set up a Latex project though, because Robin did that for us. This is something I would like to learn how to do, because I could see myself using this in future projects.
