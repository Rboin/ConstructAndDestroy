\section{Stephan Schrijver}
\subsection{Contributions}
I have contributed to this project by picking up several user stories.
The first sprint I started with the user story 'Adding buildings'. Because this game was the first C++ project I contributed to, it took me some time to get comfortable and familiar with the code base and structure. With some help of Robin I started with processing key events. First we decided that the 'N' key should trigger the game to position a castle. During the sprints we wanted to extend this by triggering this process with a panel to be able to build multiple types of buildings and actually check and subtract the resources of a player when the player builds a building.

I have implemented a generic state machine, which could be used by different classes and for different purposes. I've added the state machine to the 'Player' class so we were able to track the current state of the player and based on that we could decide what kind of action an event should trigger. I have built several states that helped us to position a building, actually add a building to the world and subtract the costs from the player its gathered resources.

Besides adding buildings, I implemented several resources like gold and stone and the entities which are able to gather these resources. Finding and editing textures, creating objects and adding them to the panels were part of implementing these resources. I also added the knight entity and the enemy version of the knight. After that was done, Sander added the behaviour of these knights.

In the last sprint we decided to trigger events like adding buildings and spawning entities through shortcut keys. First I implemented this user story without any additions to the user interface. Jeroen implemented the queue badges, and after I added more inheritance to his implementation I was able to reuse his code to display shortcut badges.

I think that I can be proud of my contribution to this project. I was afraid that, because of the programming language and my limited knowledge of it, I would become a liability. After I got more familiar with the code base I was able to pick up more tasks. Of course I asked a lot of questions but I hope it wasn't annoying, at least people didn't say so.

Compared to the other team members, I think that my contribution to the project is OK. Like I said, in the beginning I was busy studying the existing code. I felt that others were catching up easier. After the first week/first sprint I felt that I could also make some important additions.

\subsection{Qualities}
I guess the quality that I, and everyone else, have shown in this project is the fact that we did not allow bad code and bungled solutions. We all aimed for decent and neat code and were honest to each other when something could be solved on a better way.

I think the quality of my delivered code is good, but of course it could also be better. For example, I have been pointed out multiple times that deconstructors were missing in my code, which could cause memory leaks. Problems like these were fixed before the code was merged into the development branch, but it's something that I should take in mind for the next C++ project.

\subsection{Difficulties}
Some difficulties we encountered during this project were refactorings. Because the foundation of the project was developed during a short school project, some solutions that were implemented were not really durable and extendable. These problems were repaired during the first two sprints, but new features were also implemented in this time based on the old code. This caused many conflicts which took a lot of time for every team member to fix. The refactorings made the game more extendable and durable and after the conflicts were resolved, we took advantage of the refactorings.

\subsection{Lessons and learning goals}		
What I've learned during the project is that languages like C\# do a lot of stuff that you have to implement in C++ by yourself. Of course that is something that we've learned during class, but memory leaks appeared in our game, more than once. 
Developing a game in C++ gave me a clearer view of what happens 'behind the scenes' of other languages where, for example, memory management is taken care of for you.
I also gained more experience with Git. With some help of other team members I managed to rebase my code instead of merging my code with other branches. This results in a nice and clear commit history without any squashed commits or merge commits.

For future C++ projects I would like to monitor the performance and be able to track down memory issues myself. In this project Robin tracked down these problems and he told us where there might be some issues. Now I feel more confident with C++ and know were problems could appear, I want to be able to notice and then solve them, to program reliable and stable software. 

