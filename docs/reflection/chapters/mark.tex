\section{Mark van der Woude}

During the project I have worked on a lot of different parts of the project. I have worked on the following items:
\begin{itemize}
\item Project plan.
\item Selecting units. (in a rectangle)
\item Selecting one unit.
\item Selecting buildings.
\item The Player class.
\item Let the player spawn/create units.
\item Building panel.
\item Entity panel.
\item Unit panel.
\item Cost of multiple resources (visual presentation).
\item Created extra units and buildings to create a better game experience for users at Winnovation.
\item Shortcuts.
\item The description/information in the top right corner when hovering over a unit/building in the panels.
\end{itemize}

I am happy with what I have done for this project. I am happy that I worked with the UI of the project and the class structure. We have used a lot of design patterns to make the code SOLID and I think I have had a good contribution to that part. I am satisfied with all the items that I finished, I think the spawning/creating of units works really well with the factory pattern. So, that is definitely something that I am happy with. 

I do think that I should have worked a bit more on the AI part though. When items like combat goals and A* were picked up by teammates I did something else, so next time I want to pick up a bit more AI stuff. I am a bit less happy with the unit panels, although I think it looks pretty good now. During the sprint planning meeting we did not really make good decisions about what was going to be on the panel and where it would be placed. Which left us making those decisions when the parts itself were already made by different teammates.

I think the quality of my own work is good. Since I used different patterns and my code was reviewed by teammates I think most things that were not good are out of the code. Also, we used a tool to measure memory loss so I could also take out a lot of my own errors in that part to improve my code.

I am not sure which personal qualities I have showed this project. I think I have been very proactive and creative. During this project, we had a lot of freedom so we had to find items and components to work on by ourselves. I think this went very well. Each week we had some new ideas of new items we could add to the game to give the player a better experience. I think I have had a good contribution to this. 

I think we showed a lot of good qualities as a team. GitHub helped a lot with this. We had very good communication which resulted in very clear items and a good result each week. I also think this team was really helpful towards each other. For example, Robin knows a lot more about C++ than the average group member and he was always available when you needed help. Also when we worked together on items and we were not at school, we had good communication and we helped each other. This also helped because you get the same errors a lot of the time such as segmentation faults.

A pitfall for our team is that we are pretty enthusiastic, which can lead to us starting the implementation of things before each contributor knows what the end result will look like. This was a problem with the unit panel for example. There were a lot of components that needed to be done. Like the badge, name, costs, shortcuts and more. While I started on the name and the cost, Jeroen had already made the badge in the same position of the cost. That is something we should have avoided.

One of the big issues during this project were memory leaks. At first, we did not use any tools to measure memory leaks. But when we started using it, it turned out we had big memory leaks. Sander managed to fix a lot of them. After that, I checked out how he fixed that and I tried as much as possible to implement that solution into my own code to prevent making new memory leaks.

I think my contribution is very much equal to that of the rest of the team. Everybody invested the same time into to the project, except for Robin maybe who put in some extra time in the holiday. 

I learned a lot from the project about C++ and SDL. I now know a lot more about how to code C++ and what kind of errors you can expect in projects like these. Also, how to fix segmentation faults and where to look when you get errors like that. Another thing I learned is how to setup a SDL UI in C++. At first, I found the UI structure very complex and hard to work with. But once you get to know your way around it the structure is very clear and easy to work with. 

For this project, we had a really nice GitHub workflow. We used pull requests and a scrum board as well as TravisCI. I would really like to learn how to setup TravisCI myself to use it in future projects because I think that it improves the quality of the code and it is really helpful to immediately see if all the tests pass.